\documentclass[10pt]{article}
\usepackage{amsfonts}
\usepackage{fancyhdr}
\usepackage{comment}
\usepackage[letterpaper, top=2.5cm, bottom=2.5cm, left=2.2cm, right=2.2cm]%
{geometry}
\usepackage{amsmath}
\usepackage{mathtools}
\usepackage{changepage}
\usepackage{enumitem}
\usepackage{amssymb}
\usepackage{graphicx}
\usepackage{hyperref}
\usepackage{listings}
\usepackage{color}
\usepackage{textcomp}
\usepackage{courier}
\usepackage{subcaption}
\newtheorem{theorem}{Theorem}
\newtheorem{lemma}[theorem]{Lemma}
\definecolor{listinggray}{gray}{0.9}
\definecolor{lbcolor}{rgb}{0.96,0.96,0.96}
\lstset{
    backgroundcolor=\color{lbcolor},
    tabsize=4,
    rulecolor=,
    language=Python,
        basicstyle=\footnotesize\ttfamily,
        upquote=true,
        aboveskip={1.0\baselineskip},
        columns=fixed,
        extendedchars=true,
        breaklines=true,
        prebreak = \raisebox{0ex}[0ex][0ex]{\ensuremath{\hookleftarrow}},
        frame=single,
        showtabs=false,
        showspaces=false,
        showstringspaces=false,
        identifierstyle=\ttfamily,
        keywordstyle=\color[rgb]{0,0,1},
        commentstyle=\color[rgb]{0.133,0.545,0.133},
        stringstyle=\color[rgb]{0.627,0.126,0.941},
}


\begin{document}

    \title{SDS 383D, Exercises 4: Hierarchical Models: Data-analysis Problems}
    \author{Jan-Michael Cabrera}
    \date{\today}
    \maketitle

    \section*{Math tests}

    The data set in ``mathtest.csv'' shows the scores on a standardized math test from a sample of 10th-grade students at 100 different U.S.~urban high schools, all having enrollment of at least 400 10th-grade students.  (A lot of educational research involves ``survey tests'' of this sort, with tests administered to all students being the rare exception.)

    Let $\theta_i$ be the underlying mean test score for school $i$, and let $y_{ij}$ be the score for the $j$th student in school $i$.  Starting with the ``mathtest.R'' script, you'll notice that the extreme school-level averages $\bar{y}_i$ (both high and low) tend to be at schools where fewer students were sampled.

    \begin{enumerate}
        \item Explain briefly why this would be.

        

        \item Fit this normal hierarchical model to these data via Gibbs sampling:
        \begin{eqnarray*}
        y_{ij} &\sim& \mbox{N}(\theta_i, \sigma^2) \\
        \theta_i &\sim& \mbox{N}(\mu, \tau^2 \sigma^2)
        \end{eqnarray*}
        Decide upon sensible priors for the unknown model parameters $(\mu, \sigma^2, \tau^2)$.

        \item Suppose you use the posterior mean $\hat{\theta}_i$ from the above model to estimate each school-level mean $\theta_i$.  Define the shrinkage coefficient $\kappa_i$ as
        $$
        \kappa_i = \frac{ \bar{y}_i - \hat{\theta}_i}{\bar{y}_i} \, ,
        $$
        which tells you how much the posterior mean shrinks the observed sample mean.  Plot this shrinkage coefficient (in absolute value) for each school as a function of that school's sample size, and comment.

    \end{enumerate}

\end{document}