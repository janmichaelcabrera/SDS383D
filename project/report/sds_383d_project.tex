\documentclass[article]{proc}

\makeatletter
\def\BState{\State\hskip-\ALG@thistlm}
\makeatother

\begin{document}
%%%%%%%%%%%%%%%%%%%%%%%%%%%%%%%%%%%%%%%%%%%%%%%%%%%%%%%%%%%%%%%%%%%%%%%%%%%%%%%%%%%%%%%%%%%%%%%%


\PaperNumber{JCP-2019-05-10}
\title{SDS383D: Differential Flame Thermometer Uncertainty Quantification Using a Metropolis Random Walk Algorithm}

\corrauthor[1]{J.M. Cabrera}

\corremail{janmichael.cabrera@utexas.edu}

\address[1]{Department of Mechanical Engineering, The University of Texas at Austin, Austin, TX 78712}

\abstract{
    Measuring heat flux is important for characterizing fire scenarios and can be expensive to accomplish. Differential Flame Thermometers (DFT) are relatively simple, robust devices to measure heat flux but require extensive data reduction to get accurate results. Presented here is a framework for calibration of DFT's using a Cone Calorimeter and using a Metropolis Random Walk algorithm to quantify uncertainty. Two conduction models are calibrated for the insulation in the Energy Storage Method (ESM) of ASTM E3057. Both models overpredict the expected thermal conductivity at room temperature significantly. The predictor for the model error term is also quite large, approximately 20\% of the observed value, suggesting that the ESM model may not correctly characterize the physics of the problem.
}

\keywords{Directional Flame Thermometer, Heat Flux, Uncertainty Quantification, MCMC}


\maketitle
%%%%%%%%%%%%%%%%%%%%%%%%%%%%%%%%%%%%%%%%%%%%%%%%%%%%%%%%%%%%%%%%%%%%%%%%%%%%%%%%%%%%%%%%%%%%%%%%
\section{Introduction}

    Measuring the heat flux, or the amount of energy reaching a sensor per unit area per unit time, is of great import for characterization of fire scene phenomenon. However, doing so accurately is a difficult task because most sensors either do not take into account all the relevant physical effects or are not robust enough to withstand the severe environments of compartment fires. There are commercially available sensors that can be used to measure heat flux such as Gardon and Schmidt-Boelter gauges. These sensors require liquid cooling, are calibrated to measure radiative heat transfer thus creating large errors when other modes of heat transfer are important, are generally quite expensive, and also not suitable to the environments a fire scientist is interested in. 

    An official American Society for Testing and Materials (ASTM) standard was relased in 2016 for a device called the Directional Flame Thermometer (DFT) (ASTM E3057, Standard Test Method for Measuring Heat Flux Using Directional Flame Thermometers with Advanced Data Analysis Techniques~\cite{astm:3057}). The DFT is a relatively low cost device first used in the UK and later modified and adapted to the needs of Sandia National Laboratories (SNL)~\cite{Fry:1989,Nakos:2018}. The DFT is of simple construction and does not require cooling. An example of one modified for our purposes is shown in figure~\ref{fig:dft}. 

    \begin{figure}[h!]
        \centering
        \includegraphics[width=0.6\textwidth, angle=0]{../../../../../../dissertation-experimentation/media/photos/instrumentation/IMG_20190218_141904}
        \caption{Modified Differential Flame Thermometer}
        \label{fig:dft}
    \end{figure}

    The device is constructed of two 75~mm by 75~mm square by 1.6~mm thick 304 stainless steel plates, each plate instrumented on the inner surface with 24~gauge type-K thermocouples. These plates compress nominal 12~kg/m$^3$, 25.4~mm thick ceramic fiber insulation to 19.0~mm. Each plate is coated on the exterior with a high emissivity, high temperature coating. 

    While the pros of a DFT are its relative simplicity and low cost, the major con of the DFT is in reduction of sampled temperature data into an accurate representation of the incident heat flux. In contrast, Gardon and Schmidt-Boelter gauges generally have a near linear voltage response with incoming heat flux.

    In this report I introduce a quick derivation of the model used to derive the heat flux at the DFT given sampled temeprature data. I will then discuss the parameters we wish to characterize and the Bayesian framework with which this is done. Afterwards, I will discuss the use of a Metropolis Random Walk algorithm for this calibration and uncertainty quantification on two different models for our parameter of interest. Finally, I will conclude with a discussion on places in the model development and calibration methodology that we aim to improve on in the future.

\section{DFT Model}

    The first law of thermodynamics states simply that for an isolated system energy cannot be created nor destroyed. This also implies that rates of energy for a system must also be conserved. The following equation is a generalization of the first law of thermodynamics for a system exchanging energy with its surroundings:

    \begin{equation}\label{eq:energy_stored}
        \frac{dE_{sys}}{dt} = \dot{E}_{in} - \dot{E}_{out} + \dot{E}_{gen}.
    \end{equation}

    \noindent The first term represents the change in energy of the system and that must equal the rate of energy entering the system minus the energy leaving the system plus whatever energy may be generated by the system. 

    \begin{figure}[h!]
        \centering
        \includegraphics[width=0.6\textwidth, angle=0]{../../../../../../dissertation-experimentation/media/figures/diagrams/dft_energy_balance}
        \caption{Energy balance on a DFT}
        \label{fig:dft_energy}
    \end{figure}

    Figure~\ref{fig:dft_energy} shows a DFT with a control volume (dotted line) showing the boundaries of the system we are interested in analyzaing. We begin by noting the sources of incoming and outgoing energy for the system:

    \begin{align}\label{eq:dft_balance}
        \dot{E}_{in} &= q_{inc,r} \\
        \dot{E}_{out} &= q_{refl} + q_{emit} + q_{conv} \\
        \dot{E}_{gen} & = 0,
    \end{align}

    \noindent where $q_{inc,r}$ is the incoming incident radiative heat transfer (the quantity we are interested in predicting), and $q_{refl}$, $q_{emit}$ and $q_{conv}$ are the reflected radiative heat transfer, emitted radiative heat transfer from the DFT's surface, and convective heat transfer respectively. The energy generation term is zero because there are no chemical reactions taking place that could generate or consume energy.

    The change of energy of our system, $\frac{dE_{sys}}{dt}$, is defined here as $q_{net}$ and reduces to the following:

    \begin{align}\label{eq:q_net}
        q_{net} &= \dot{E}_{in} - \dot{E}_{out} \\
            &= q_{inc,r} - q_{refl} - q_{emit} - q_{conv}.
    \end{align}

    Up until now, no approximations have been made for the analysis of the DFT. We now need to introduce various well established approximations to the heat transfer models for each of the terms in equation~\ref{eq:q_net}. The emitted radiative heat transfer from the DFT is

    \begin{equation}\label{eq:q_emit}
        q_{emit} = \sigma \varepsilon_f (T_f^4 - T_{sur}^4),
    \end{equation}

    \noindent where $\sigma$ is the Stefan-Boltzman constant, $\varepsilon_f$ is the emissivity of the front (top) of the DFT, $T_f$ is the front (top) temperature of the DFT, and $T_{sur}$ is the temperature of the surroundings far away from the DFT to which radiative losses are transfered. The reflected radiative heat transfer is simply one minus the absorped radiative heat transfer:

    \begin{equation}\label{eq:q_refl}
        q_{refl} = (1 - \alpha_f)q_{inc}.
    \end{equation}

    \noindent Here $\alpha_f$ is the absorptivity of the front (top) of the DFT. Convective losses from the DFT follow from Newton's Law of Cooling:

    \begin{equation}\label{eq:q_conv}
        q_{conv} = h (T_f - T_{\infty}),
    \end{equation}

    \noindent where $h$ is the heat transfer coefficient, $T_f$ is the front (top) temperature of the DFT, and $T_{\infty}$ is defined as the temperature of the surrounding fluid 'far' from the DFT. Combining equations~\ref{eq:q_emit} through \ref{eq:q_conv} with equation~\ref{eq:q_net}, asuming a thermally grey surface (i.e. $\alpha_f = \varepsilon_f$), and solving for $q_{inc,r}$ we get,

    \begin{equation}\label{eq:q_inc}
        q_{inc,r} = (q_{net}/\varepsilon_f) + \sigma (T_f^4 - T_{surf}^4) + h/\varepsilon_f (T_f - T_{\infty}).
    \end{equation}

    What is left is to model $q_{net}$ the energy stored within the DFT. A full derivation is beyond the scope of this report so we choose to use the simplified equation presented in ASTM E3057 as the Energy Storage Method (ESM):

    \begin{equation}\label{eq:esm}
        q_{net} = \left(\rho_s c_{p,s}(T) l_s \frac{dT_{DFT}}{dt} \right) + \left( k_{ins}(T) \frac{T_f - T_r}{l_{ins}} \right) + \left(\rho_{ins} c_{p,ins}(T) l_{ins} \frac{dT_{ins}}{dt}\right).
    \end{equation}

    \noindent The first term on the right hand side represents the energy stored in the top plate of the DFT were $\rho_s$, $c_{p,s}(T)$, and $l_s$ are the density, constant pressure specific heat, and thickness of the plate respectivetly. The second term represents energy lost through the insulation to the back plate where $k_{ins}(T)$ is the thermal conducitivity of the insulation, and the third term is the energy stored within the insulation where $\rho_{ins}$, $c_{p,ins}(T)$, and $l_{ins}$ are the density, constant pressure specific heat, and thickness of the insulation respectively. 

    The thermal properties of 304 stainless steel are well characterized within the literature. The density and specific heat of the insulation are also fairly well known. However, the thermal conductivity of the compressed insulation is difficult to accurately characterize. In calibration of the incident heat flux to the DFT, we choose to characterize two different models for the thermal conductivity of the insulation. The first model assumes that there is no temperature dependence,

    \begin{equation}\label{eq:k_ins_1}
        k_{ins}^{(1)}(T) = k_0^{(1)}.
    \end{equation}

    \noindent The second model assumes that there is a linear dependence on temperature for the thermal conductivity of the insulation,

    \begin{equation}\label{eq:k_ins_2}
        k_{ins}^{(2)}(T) = k^{(2)}_0 + k^{(2)}_1 T.
    \end{equation}

    In order to calibrate these models, observed data for the incident heat flux needed to be taken. The following section summarizes the calibration procedure for obtaining these data.
    
\section{Calibration Procedure}

    Calibration of an instrument is generally done by referencing a more accurate instrument. The DFT's were calibrated against a Schmidt-Boelter gauge that is accurate to within $\pm 3\%$ of the reading. The radiant element in an ASTM E1354 Cone Calorimeter was used to produce the necessary incident heat flux on both the Schmidt-Boelter gauge and the DFT~\cite{conecal}. Figure~\ref{fig:cone_dft} shows the DFT beneath the heating element. 

    \begin{figure}[h!]
        \centering
        \includegraphics[width=0.6\textwidth, angle=-90]{../../../../../../dissertation-experimentation/media/photos/instrumentation/IMG_20190409_090942}
        \caption{Cone calorimeter for calibration of DFT}
        \label{fig:cone_dft}
    \end{figure}

    The Schmidt-Boelter gauge was placed with its top surface 25~mm from the bottom of the heating elements and the cone temperature was set such that the incident heat flux on the gauge was 5~kW/m$^2$. The two thermocouples on the DFT were connected to a temperature logger. The DFT was left outside of the heating element of the Cone Calorimeter and one minute of pre-test data was collected. The radiation shield on the Cone Calorimeter was put in place and the DFT was moved onto the platform such that when one minute had elapsed, the radiation shield would be opened allowing the DFT to receive the heat flux from the heating element. The DFT was also placed such that the front plate was 25~mm away from the bottom of the heating elements. The DFT was exposed to the heat flux for five minutes at which point the radiation shield was re-engaged, the DFT moved away from the Cone Calorimeter, and a further minute of post-test data was taken. In total seven minutes of data was taken. This process was repeated for each of 20 DFTs made, however for brevity, the data of one test will be presented. 

    The following section introduces the parameter calibration framework used given the known incident heat flux measured by the Schmidt-Boelter gauge and the temperatures recorded from the DFT. 

\section{Parameter Calibration}

    The parameter calibration process was completed in two different ways for each model. The first involved finding the parameters for the model that minimized the sum of the squared errors between the observed heat flux and the predicted heat flux from the model. The second involved construction of the conditional distributions for the parameters of interest using a Metropolis Random Walk algorithm. 

    \subsection{Minimizing Residual Sum of Squared Errors}

        Finding the parameters that minimizes the error between the predicted heat flux and the observed heat flux is a fairly straightforward process that is depicted by

        \begin{equation}\label{eq:rss}
            \hat{\theta}_{RSS} = \arg \min_{\theta}  \sum_{i=1}^n \left(q_i^{(inc,r)} - \hat{q}_i^{(inc,r)} \right)^2.
        \end{equation}

        \noindent A loss function is defined as the summation term on the right hand side of the equation. A Sequential Least Squares Programming (SLSQP) algorithm from the Python SciPy package was used to obtain $\hat{\theta}_{RSS}$. The results from this optimization step were saved and used to initiate the chains of the Metropolis Random Walk algorithm discussed later. 

        % $$\sigma_{xy} \leq 0.9 \sigma_x \sigma_y$$

    \subsection{Full Conditional Distributions}

        The conditional distribution for our parameters of interest given Bayes' Rule is the following

        \begin{align}\label{eq:bayes}
            P(\theta | D, \sigma_q^2, M_k) &= P(\theta|M_k) \frac{P(D|\theta, \sigma_q^2, M_k)}{P(D|M_k)}\\
                &\propto P(\theta|M_k) P(D|\theta, \sigma_q^2, M_k),
        \end{align}

        \noindent Where $D$ represents the set of $n$ heat flux observations during the course of a test ($D = \{q^{(inc,r)}_1, \dots, q^{(inc,r)}_n \}$), $M_k$ represents a particular model used to generate a set of $n$ predictions $\hat{q}^{(inc,r)}_i$, and $\sigma_q^2$ is the assumed error between the model predictions and the observed heat fluxes. Here we choose to adopt a Gaussian error model between the predicted and observed heat fluxes,

        \begin{equation}\label{eq:model}
            q^{(inc,r)}_i = \hat{q}^{(inc,r)}_i + e_q; \hspace{10pt} e_q \sim N(0, \sigma_q^2). 
        \end{equation}

        \noindent Explicitly, the Gaussian error model for a single observation takes the form of, 

        \begin{equation}\label{eq:likelihood}
            P(q_i^{(inc,r)}| \hat{q}_i^{(inc,r)}, \sigma_q^2, M_k) = \left(\frac{1}{2 \pi \sigma_q^2} \right)^{1/2} \text{exp} \left[-\frac{1}{2 \sigma_q^2} \left( q_i^{(inc,r)} - \hat{q}_i^{(inc,r)} \right)^2 \right].
        \end{equation}

        \noindent The resulant likelihood for the set of $n$ independent observations is then,

        \begin{align}\label{eq:likelihood_2}
            P(D| \theta, \sigma_q^2, M_k) &= \prod_{i=1}^n P(q_i^{(inc,r)}| \hat{q}_i^{(inc,r)}, \sigma_q^2) \\
            &= \left(\frac{1}{2 \pi \sigma_q^2} \right)^{n/2} \text{exp} \left[-\frac{1}{2 \sigma_q^2} \sum_{i=1}^n \left(q_i^{(inc,r)} - \hat{q}_i^{(inc,r)} \right)^2 \right].
        \end{align}

        \noindent Because there is no means to find a conjugate prior for $\theta$ with which to get an analytically closed form to the conditional distribution for this given problem, a Metropolis Random Walk algorithm was used to sample from the conditional distribution.

        Derivation of the conditional distribution for the model error term follows a similar logic:

        \begin{equation}\label{eq:sigma_q}
            P(\sigma_q^2 | D, \theta, M_k) \propto P(\sigma_q^2) P(D| \theta, \sigma_q^2, M_k).
        \end{equation}

        \noindent Here we choose to use an uninformative improper prior (i.e. $P(\sigma_q^2) \propto 1/\sigma_q^2$) and derive an analytical form of the conditional distribution. In this case the resultant conditional distribution will take the form of an Inverse-Gamma distribution parameterized by the following,

        \begin{equation}\label{eq:sigma_q_conditional}
            \sigma_q^2 | D, \theta, M_k \sim IG \left(\frac{n}{2},  \frac{1}{2}\sum_{i=1}^n \left(q_i^{(inc,r)} - \hat{q}_i^{(inc,r)} \right)^2\right).
        \end{equation}

        \noindent A Gibbs sampling step is performed to sample from the conditional distribution above. Details of the Metropolis Random Walk algorithm are presented in the next section.

    \subsection{Metropolis Random Walk}

        A Metropolis Random Walk algorithm was written in Python in order to sample from the conditional distributions described in the previous section. The psuedo-code is summarized in algorithm~\ref{alg:metropolis}. The parameters of interest are initialized with the values found using equation~\ref{eq:rss}.   

        \begin{algorithm}
        \caption{Metropolis Random Walk}\label{alg:metropolis}
        \begin{algorithmic}[1]
        \State Initialize $\theta^{(0)}$
        \State Initialize $\sigma_e^2$
        \State Initialize $i = 1$
        \While {$a < Samples$}
            \State Propose: $\theta^{\star} = \theta^{(i)} + e, \hspace{10pt} e \sim N(0, \sigma_e^2)$
            \State Acceptance Probability: $$\beta (\theta^{\star} | \theta^{(i-1)}) = \frac{P(\theta^{\star})P(D|\theta^{\star},\sigma_{q,i,k}^2,M_k)}{P(\theta^{(i-1)})P(D|\theta^{(i-1)},\sigma_{q,i,k}^2, M_k)}$$
                \If {Uniform(0,1) $< \text{min} \left\{1, \beta (\theta^{\star} | \theta^{(i-1)})\right \}$}
                    \State Accept Proposal: $\theta^{(i)} = \theta^{\star}$
                    \State $\sigma_{q,i,k}^2 | D, \theta, M_k \sim IG \left(\frac{n}{2},  \frac{1}{2}\sum_{i=1}^n \left(q_i^{(inc,r)} - \hat{q}_i^{(inc,r)} \right)^2\right)$
                    \State $a+=1$
                \Else
                    \State Reject Proposal: $\theta^{(i)} = \theta^{(i-1)}$
                \EndIf
            \While {$Tuning$}
                \State $\sigma_e^2 = 2.4^2 Var(\theta^{(a-tune)}, \dots, \theta^{(a)}|D)/d$
            \EndWhile
            \State $i+=1$
        \EndWhile
        \end{algorithmic}
        \end{algorithm}

        While the number of accepted samples is less than the number of prescribed samples plus the tuning samples, the algorithm samples from the conditional distribution for $\theta$. A new value for theta, $\theta^{\star}$, is proposed by sampling from a Normal proposal distribution. The acceptance probability, $\beta$, is calculated and it is accepted if the minimum value between $\beta$ and one is greater than a uniform random draw between zero and one.

        During tuning of the proposal covariance, the previous $t$ accepted samples are used to update the covariance of the proposal. Tuning occurs $T$ times before the algorithm is allowed to collect samples from the conditional distributions. Tuning of the covariance follows that presented in Niemi~\cite{Niemi:2019}. 

\section{Results}

    Results from the experiments and the parameter calibration for the two conduction models are presented in the following subsections. The data reduction section discusses the need for smoothing of the data, nd the parameter calibration section discusses the results from the fits. 

    \subsection{Data Reduction}

        It was noticed that there was significant bit noise in the temperature resulting in a stair-step-like function, left of figure~\ref{fig:tc_data}. Taking the gradient of a noisy signal such as this will result in unrealistic fluctuations of the predicted heat flux. For this reason the temperature data was smoothed.

        \begin{figure}[!]
            \centering
            \subfigure{\includegraphics[width=0.4\textwidth]{../data/unsmoothed/5_kw_m2/plots/1903-02_05}}
            \qquad
            \subfigure{\includegraphics[width=0.4\textwidth]{../data/smoothed/5_kw_m2/plots/1903-02_05}}
            \caption{Unsmoothed and smoothed temperature data for one DFT test}
            \label{fig:tc_data}
        \end{figure}

        The temperature data was smoothed using Gaussian procces regression with a Matern (5,2) covariance function, and optimzed using the marginal likelihood~\cite{Rasmussen:2006}. The resultant smoothed data is shown on the right of figure~\ref{fig:tc_data}. Here we choose to only use the smoothed data and ignore the resultant uncertainty in the fit since we will be passing these through a function to determine the predicted heat flux.

        Given the smoothed temperature, the models described by equations~\ref{eq:q_inc} through \ref{eq:k_ins_1}, and equations~\ref{eq:q_inc}, \ref{eq:esm}, and \ref{eq:k_ins_2} are calibrated. Results of the calibration are presented in the next section.

        % \begin{figure}
        %     \centering
        %     \includegraphics[width=0.6\textwidth]{../data/residuals/5_kw_m2/1903-02_05}
        %     \caption{Residuals from the Gaussian process fit}
        %     \label{fig:gp_residuals}
        % \end{figure}

    \subsection{Parameter Calibration}

        \subsubsection{Model One - $k^{(1)}_{ins}(T) = k^{(1)}_0$:}

        Twenty-thousand samples from the conditional distribution for model one were sampled using the Metropolis Random Walk Algorithm after a total of 1000 tuning steps. The resultant ratio of accepted to total samples of 0.41 was close to the optimal of 0.45, described in Roberts et al~\cite{Roberts:1997}.

        \begin{figure}[b!]
            \centering
            \subfigure{\includegraphics[width=0.4\textwidth]{../figures/alpha_hist_0_dft_5_kwm2_1}}
            \qquad
            \subfigure{\includegraphics[width=0.4\textwidth]{../figures/alpha_trace_0_dft_5_kwm2_1}}
            \caption{Histogram and trace plots for thermal conductivity of insulation, $k^{(1)}_{ins}$}
            \label{fig:param_trace_1}
        \end{figure}

        \begin{figure}[b!]
            \centering
            \subfigure{\includegraphics[width=0.4\textwidth]{../figures/sigma_hist_dft_5_kwm2_1}}
            \qquad
            \subfigure{\includegraphics[width=0.4\textwidth]{../figures/sigma_trace_dft_5_kwm2_1}}
            \caption{Histogram and trace plots for $\sigma_{q,(1)}^2$}
            \label{fig:sigma_trace_1}
        \end{figure}

        Parameter histograms and traces are presented in figures~\ref{fig:param_trace_1} and \ref{fig:sigma_trace_1} for the thermal conductivity and model error terms respectively. The predicted results from the model are shown in figure~\ref{fig:cal_results_1}. In this figure, the solid black line represents the "observed" heat flux, the solid red-line represents a prediction using a legacy high-order polynomial fit for the temperature response of the insulation thermal conductivity, the solid blue line represents the prediction given the mean of the distribution shown in figure~\ref{fig:param_trace_1}, and finally, the dashed green line represents the optimal fit from minimizing the residual sum of squared errors. The grey band on the plot represents realizations of the prediction for each sample from the parameter trace in figure~\ref{fig:param_trace_1}. 

        \begin{figure}[!]
            \centering
            \includegraphics[width=0.8\textwidth]{../figures/calibrated_results_dft_5_kwm2_1}
            \caption{Calibrated results assuming $k^{(1)}_{ins}(T) = k^{(1)}_0$}
            \label{fig:cal_results_1}
        \end{figure}

        \subsubsection{Model Two - $k^{(2)}_{ins}(T) = k^{(2)}_0 + k^{(2)}_1 T$:}

        Five-thousand samples from the conditional distirbutions for model two were sampled using the Metropolis Random Walk algorithm after a total of 1000 tuning steps. The resultant ratio of accepted to total samples was 0.02. Note that as the dimension of the sampler increases the optimal ratio asymptotically approaches 0.234~\cite{Roberts:1997}. 

        \begin{figure}[b!]
            \centering
            \subfigure{\includegraphics[width=0.4\textwidth]{../figures/alpha_hist_0_dft_5_kwm2_2}}
            \qquad
            \subfigure{\includegraphics[width=0.4\textwidth]{../figures/alpha_trace_0_dft_5_kwm2_2}}

            \subfigure{\includegraphics[width=0.4\textwidth]{../figures/alpha_hist_1_dft_5_kwm2_2}}
            \qquad
            \subfigure{\includegraphics[width=0.4\textwidth]{../figures/alpha_trace_1_dft_5_kwm2_2}}
            \caption{Histogram and trace plots for thermal conductivity of insulation, $k^{(2)}_{ins}$. Top is $k^{(2)}_0$ and bottom is $k^{(2)}_1$.}
            \label{fig:param_trace_2}
        \end{figure}

        Parameter histograms and traces are presented in figures~\ref{fig:param_trace_2} and \ref{fig:sigma_trace_2} for the thermal conductivity and model error terms respectively. Figure~\ref{fig:alpha_correlation} shows the correlation between the samples from the parameter traces shown in figure~\ref{fig:param_trace_2}. 

        \begin{figure}[h!]
            \centering
            \includegraphics[width=0.6\textwidth]{../figures/alpha_correlation}
            \caption{Scatter plot showing the correlation between $k^{(2)}_0$ and $k^{(2)}_1$.}
            \label{fig:alpha_correlation}
        \end{figure}

        The predicted results from the model are shown in figure~\ref{fig:cal_results_2}. Again, here the solid black line represents the "observed" heat flux, the solid red-line represents a prediction using a legacy high-order polynomial fit for the temperature response of the insulation thermal conductivity, the solid blue line represents the prediction given the mean of the distributions shown in figure~\ref{fig:param_trace_2}, and finally, the dashed green line represents the optimal fit from minimizing the residual sum of squared errors. The grey band on the plot represents realizations of the prediction for each sample from the parameter traces in figure~\ref{fig:param_trace_2}.

        A plot of the traces with coefficient zero on the x-axis and coefficient one on the y-axis for the second model is shown in figure~\ref{fig:alpha_correlation}. It is surprising how strongly negatively correlated the coefficients are despite the sampling covariance for the proposal distribution of the Metropolis Random Walk algorithm only containing diagonal elements (off diagonals were zero). The resultant correlation between the two coefficients was -0.9993. 

        \begin{figure}[!]
            \centering
            \subfigure{\includegraphics[width=0.4\textwidth]{../figures/sigma_hist_dft_5_kwm2_2}}
            \qquad
            \subfigure{\includegraphics[width=0.4\textwidth]{../figures/sigma_trace_dft_5_kwm2_2}}
            \caption{Histogram and trace plots for $\sigma_{q,(2)}^2$}
            \label{fig:sigma_trace_2}
        \end{figure}

        \begin{figure}[!]
            \centering
            \includegraphics[width=0.8\textwidth]{../figures/calibrated_results_dft_5_kwm2_2}
            \caption{Calibrated results assuming $k^{(2)}_{ins}(T) = k^{(2)}_0 + k^{(2)}_1 T$}
            \label{fig:cal_results_2}
        \end{figure}

        \begin{table}[!]
        \centering
        \caption{Calibration results for models one and two}
        \begin{tabular}{ccc}
            \hline
            - & mean & standard deviation \\
            \hline
            $k_0^{(1)}$ (W/m$\cdot$K)       & 0.312     & 0.024 \\
            $\sigma_{q, (1)}^2$ (kW/m$^2$)  & 0.980     & 0.067 \\
            \hline
            $k_0^{(2)}$ (W/m$\cdot$K)       & -3.356    & 0.730 \\
            $k_{1}^{(2)}$ (W/m$\cdot$K)     & 0.011     & 0.002 \\
            $\sigma_{q, (2)}^2$ (kW/m$^2$)  & 0.920     & 0.065 \\
            \hline
        \end{tabular}
        \label{tbl:cal_results}
        \end{table}

        Results from the two calibrations are summarized in table~\ref{tbl:cal_results}. The coefficients for the second model are difficult to interpret directly. Over the course of an observation, the temperature of the insulation varies from roughly 290~K to 350~K. The resultant $k_{ins}^{(2)(T)}$ varies from -0.1 to 0.6~W/m$\cdot$K. Negative values here are not physical. Both models overpredict the thermal conductivity at higher temperatures and the second model displays an unrealistic dependence on temperature. 

        The predictors for the model error term for each model are approximately 20\% that of the incident heat flux. This suggests that the ESM model may not be appropriate here for our modified DFTs.

\section{Conclusions}

    We presented a framework for the calibration of DFT's using a Cone Calorimeter and a Metropolis Random Walk algorithm for uncertainty quantification. Two conduction models were calibrated for the insulation in the ESM described in ASTM E3057 for modified DFT's. Both models showed unphysical values for the thermal conductivity and the second of the two models showed an unrealistic dependence on temperature. 

    Much of the modeling error here was likely caused by the assumptions made in the use of the ESM model. The model only takes into account one dimensional conductive heat transfer through the insulation and also tends to overestimate the energy stored within the insulation. It is likely that because the size of the modified DFTs was smaller than that presented in the standard, conduction through the fasteners holding the DFT's together could not be ignored. 

    Another area of improvement would be the assumption of the curve for the "observed" incident heat flux. Work is being done to obtain a "true" incident heat flux as measured by the Schmidt-Boelter gauge. Further work is being developed to more accurately model the heat transfer through the insulation as well as to account for conduction through the bolts of the DFT.
    
\clearpage
%% References with bibTeX database:
\bibliographystyle{Bibliography_Style}
\scriptsize{
\bibliography{../../../../../../dissertation-experimentation/References}
}

\end{document}
