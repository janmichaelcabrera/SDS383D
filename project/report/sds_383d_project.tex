\documentclass[article]{proc}

\makeatletter
\def\BState{\State\hskip-\ALG@thistlm}
\makeatother

\begin{document}
%%%%%%%%%%%%%%%%%%%%%%%%%%%%%%%%%%%%%%%%%%%%%%%%%%%%%%%%%%%%%%%%%%%%%%%%%%%%%%%%%%%%%%%%%%%%%%%%


\PaperNumber{JCP-2019-05-10}
\title{SDS383D: Differential Flame Thermometer Uncertainty Quantification Using a Metropolis Random Walk Algorithm}

\corrauthor[1]{J.M. Cabrera}

\corremail{janmichael.cabrera@utexas.edu}

\address[1]{Department of Mechanical Engineering, The University of Texas at Austin, Austin, TX 78712}

\abstract{
Some Text
}

\keywords{Directional Flame Thermometer, Heat Flux, Uncertainty Quantification, MCMC}


\maketitle
%%%%%%%%%%%%%%%%%%%%%%%%%%%%%%%%%%%%%%%%%%%%%%%%%%%%%%%%%%%%%%%%%%%%%%%%%%%%%%%%%%%%%%%%%%%%%%%%
\section{Introduction}

    Measuring the heat flux, or the amount of energy reaching a sensor per unit area per unit time, is of great import for characterization of fire scene phenomenon. However, doing so accurately is a difficult task because most sensors either do not take into account all the relevant physical effects or are not robust enough to withstand the severe environments of compartment fires. There are commercially available sensors that can be used to measure heat flux such as Gardon and Schmidt-Boelter gauges. These sensors require liquid cooling, are calibrated to measure radiative heat transfer thus creating large errors when other modes of heat transfer are important, are generally quite expensive, and also not suitable to the environments a fire scientist is interested in. 

    An official American Society for Testing and Materials (ASTM) standard was relased in 2016 (ref XXX) for a device called the Directional Flame Thermometer (DFT) (ASTM E3057, Standard Test Method for Measuring Heat Flux Using Directional Flame Thermometers with Advanced Data Analysis Techniques). The DFT is a relatively low cost device first used in the UK (ref XXX) and later modified and adapted to the needs of Sandia National Laboratories (SNL) (ref XXX). The DFT is of simple construction and does not require cooling. An example of one modified for our purposes is shown in figure~\ref{fig:dft}. 

    \begin{figure}[h!]
        \centering
        \includegraphics[width=0.6\textwidth, angle=0]{../../../../../../dissertation-experimentation/media/photos/instrumentation/IMG_20190218_141904}
        \caption{Modified Differential Flame Thermometer}
        \label{fig:dft}
    \end{figure}

    The device is constructed of two 75~mm by 75~mm square by 1.6~mm thick 304S stainless steel plates, each plate instrumented on the inner surface with 24~gauge type-K thermocouples. These plates compress nominal 12~kg/m$^3$, 25.4~mm thick ceramic fiber insulation to 19.0~mm. Each plate is coated on the exterior with a high emissivity, high temperature coating. 

    While the pros of a DFT are its relative simplicity and low cost, the major con of the DFT is in reduction of sampled temperature data into an accurate representation of the incident heat flux. In contrast, Gardon and Schmidt-Boelter gauges generally have a near linear voltage response with incoming heat flux.

    In this report I while introduce a quick derivation of the model used to derive the heat flux at the DFT given sampled temeprature data. I will then discuss the parameters we wish to characterize and the Bayesian framework with which this is done. Afterwards, I will discuss the use of a Metropolis Random Walk algorithm for this calibration and uncertainty quantification on two different models for our parameter of interest. Finally, I will conclude with a discussion on places in the model development and calibration methodology that we aim to improve on in the future.

\section{DFT Model}

    The first law of thermodynamics states simply that for an isolated system energy cannot be created nor destroyed. This also implies that rates of energy for a system must also be conserved. The following equation is a generalization of the first law of thermodynamics for a system exchanging energy with its surroundings:

    \begin{equation}\label{eq:energy_stored}
        \frac{dE_{sys}}{dt} = \dot{E}_{in} - \dot{E}_{out} + \dot{E}_{gen}.
    \end{equation}

    \noindent The first term represents the change in energy of the system and that must equal the rate of energy entering the system minus the energy leaving the system plus whatever energy may be generated by the system. 

    \begin{figure}[h!]
        \centering
        \includegraphics[width=0.6\textwidth, angle=0]{../../../../../../dissertation-experimentation/media/figures/diagrams/dft_energy_balance}
        \caption{Energy balance on a DFT}
        \label{fig:dft_energy}
    \end{figure}

    Figure~\ref{fig:dft_energy} shows a DFT with a control volume (dotted line) showing the boundaries of the system we are interested in analyzaing. We begin by noting the sources of incoming and outgoing energy for the system:

    \begin{align}\label{eq:dft_balance}
        \dot{E}_{in} &= q_{inc,r} \\
        \dot{E}_{out} &= q_{refl} + q_{emit} + q_{conv} \\
        \dot{E}_{gen} & = 0,
    \end{align}

    \noindent where $q_{inc,r}$ is the incoming incident radiative heat transfer (the quantity we are interested in predicting), and $q_{refl}$, $q_{emit}$ and $q_{conv}$ are the reflected radiative heat transfer, emitted radiative heat transfer from the DFT's surface, and convective heat transfer respectively. The energy generation term is zero because there are no chemical reactions taking place that could generate or consume energy.

    The change of energy of our system, $\frac{dE_{sys}}{dt}$, is defined here as $q_{net}$ and reduces to the following:

    \begin{align}\label{eq:q_net}
        q_{net} &= \dot{E}_{in} - \dot{E}_{out} \\
            &= q_{inc,r} - q_{refl} - q_{emit} - q_{conv}.
    \end{align}

    Up until now, no approximations have been made for the analysis of the DFT. We now need to introduce various well established approximations to the heat transfer models for each of the terms in equation~\ref{eq:q_net}. The emitted radiative heat transfer from the DFT is

    \begin{equation}\label{eq:q_emit}
        q_{emit} = \sigma \varepsilon_f (T_f^4 - T_{sur}^4),
    \end{equation}

    \noindent where $\sigma$ is the Stefan-Boltzman constant, $\varepsilon_f$ is the emissivity of the front (top) of the DFT, $T_f$ is the front (top) temperature of the DFT, and $T_{sur}$ is the temperature of the surroundings far away from the DFT to which radiative losses are transfered. The reflected radiative heat transfer is simply one minus the absorped radiative heat transfer:

    \begin{equation}\label{eq:q_refl}
        q_{refl} = (1 - \alpha_f)q_{inc}.
    \end{equation}

    \noindent Here $\alpha_f$ is the absorptivity of the front (top) of the DFT. Convective losses from the DFT follow from Newton's Law of Cooling:

    \begin{equation}\label{eq:q_conv}
        q_{conv} = h (T_f - T_{\infty}),
    \end{equation}

    \noindent where $h$ is the heat transfer coefficient, $T_f$ is the front (top) temperature of the DFT, and $T_{\infty}$ is defined as the temperature of the surrounding fluid 'far' from the DFT. Combining equations~\ref{eq:q_emit} through \ref{eq:q_conv} with equation~\ref{eq:q_net}, asuming a thermally grey surface (i.e. $\alpha_f = \varepsilon_f$), and solving for $q_{inc,r}$ we get,

    \begin{equation}\label{eq:q_inc}
        q_{inc,r} = (q_{net}/\varepsilon_f) + \sigma (T_f^4 - T_{surf}^4) + h/\varepsilon_f (T_f - T_{\infty}).
    \end{equation}

    What is left is to model $q_{net}$ the energy stored within the DFT. A full derivation is beyond the scope of this report. The equation used is in reference to the Energy Storage Method (ESM) outlined in ASTM E3057:

    \begin{equation}\label{eq:esm}
        q_{net} = \left(\rho_s c_{p,s}(T) l_s \frac{dT_{DFT}}{dt} \right) + \left( k_{ins}(T) \frac{T_f - T_r}{l_{ins}} \right) + \left(\rho_{ins} c_{p,ins}(T) l_{ins} \frac{dT_{ins}}{dt}\right).
    \end{equation}

    \noindent The first term on the right hand side represents the energy stored in the top plate of the DFT were $\rho_s$, $c_{p,s}(T)$, and $l_s$ are the density, constant pressure specific heat, and thickness of the plate respectivetly. The second term represents energy lost through the insulation to the back plate where $k_{ins}(T)$ is the thermal conducitivity of the insulation, and the third term is the energy stored within the insulation where $\rho_{ins}$, $c_{p,ins}(T)$, and $l_{ins}$ are the density, constant pressure specific heat, and thickness of the insulation respectively. 

    The thermal properties of 304S stainless steel is well characterized within the literature. The density and specific heat of the insulation is also fairly well known. However, the thermal conductivity of the compressed insulation is difficult to accurately characterize. In calibration of the incident heat flux to the DFT, we choose to characterize two different models for the thermal conductivity of the insulation. The first model assumes that there is no temperature dependence,

    \begin{equation}\label{eq:k_ins_1}
        k_{ins}^{(1)}(T) = k_0^{(1)}.
    \end{equation}

    \noindent The second model assumes that there is a linear dependence on temperature for the thermal conductivity of the insulation,

    \begin{equation}\label{eq:k_ins_2}
        k_{ins}^{(2)}(T) = k^{(2)}_0 + k^{(2)}_1 T.
    \end{equation}

    In order to calibrate these models, observed data for the incident heat flux needed to be taken. The following section summarizes the calibration procedure for obtaining these data.
    
\section{Calibration Procedure}

    Calibration of an instrument is generally done by referencing a more accurate instrument. The DFT's were calibrated against a Schmidt-Boelter gauge that is accurate to within $\pm 3\%$ of the reading. The radiant element in an ASTM E1354 Cone Calorimeter was used to produce the necessary incident heat flux on both the Schmidt-Boelter gauge and the DFT. Figure~\ref{fig:cone_dft} shows the DFT beneath the heating element. 

    \begin{figure}[h!]
        \centering
        \includegraphics[width=0.6\textwidth, angle=-90]{../../../../../../dissertation-experimentation/media/photos/instrumentation/IMG_20190409_090942}
        \caption{Cone calorimeter for calibration of DFT}
        \label{fig:cone_dft}
    \end{figure}

    The Schmidt-Boelter gauge was placed with its top surface 25~mm from the bottom of the heating elements and the cone temperature was set such that the incident heat flux on the gauge was 5~kW/m$^2$. The two thermocouples on the DFT were connected to a temperature logger and one minute of pre-test data was collected. The radiation shield on the Cone Calorimeter was put in place and the DFT was moved onto the platform such that when one minute had elapsed, the radiation shield would be opened allowing the DFT to receive the heat flux from the heating element. The DFT was also placed such that the front plate was 25~mm away from the bottom of the heating elements. The DFT was exposed to the heat flux for five minutes at which point the radiation shield was re-engaged and a further minute of post-test data was taken. In total seven minutes of data was taken. This process was repeated for each of the 20 DFTs made, however for brevity, the data of one test will be presented. 

    The following section introduces the parameter calibration framework used given the known incident heat flux measured by the Schmidt-Boelter gauge and the temperatures recorded from the DFT. 

    \section{Parameter Calibration}

    \begin{align}\label{eq:bayes}
        P(\theta | D, M) &= P(\theta|M) \frac{P(D|\theta, M)}{P(D|M)}\\
            &\propto P(\theta|M) P(D|\theta, M)
    \end{align}

    \noindent $D = \{q^{(inc,r)}_1, \dots, q^{(inc,r)}_n \}$

    \begin{equation}\label{eq:likelihood}
        P(q_i^{(inc,r)}| \hat{q}_i^{(inc,r)}, \sigma_q^2) = \left(\frac{1}{2 \pi \sigma_q^2} \right)^{1/2} \text{exp} \left[-\frac{1}{2 \sigma_q^2} \left( q_i^{(inc,r)} - \hat{q}_i^{(inc,r)} \right)^2 \right]
    \end{equation}

    \begin{align}\label{eq:likelihood_2}
        P(D| \theta, M) &= \prod_{i=1}^n P(q_i^{(inc,r)}| \hat{q}_i^{(inc,r)}, \sigma_q^2) \\
        &= \left(\frac{1}{2 \pi \sigma_q^2} \right)^{n/2} \text{exp} \left[-\frac{1}{2 \sigma_q^2} \sum_{i=1}^n \left(q_i^{(inc,r)} - \hat{q}_i^{(inc,r)} \right)^2 \right]
    \end{align}

\section{Metropolis Random Walk}

\begin{algorithm}
\caption{Metropolis Random Walk}\label{euclid}
\begin{algorithmic}[1]
\State Initialize $\theta^{(0)}$
\State Initialize $\sigma_e^2$
\State Initialize $i = 1$
\While {$a < Samples$}
    \State Propose: $\theta^{\star} = \theta^{(i)} + e, \hspace{10pt} e \sim N(0, \sigma_e^2)$
    \State Acceptance Probability: $$\beta (\theta^{\star} | \theta^{(i-1)}) = \frac{P(\theta^{\star})P(D|\theta^{\star},M)}{P(\theta^{(i-1)})P(D|\theta^{(i-1)}, M)}$$
        \If {Uniform(0,1) $< \text{min} \left\{1, \beta (\theta^{\star} | \theta^{(i-1)})\right \}$}
            \State Accept Proposal: $\theta^{(i)} = \theta^{\star}$
            \State $a+=1$
        \Else
            \State Reject Proposal: $\theta^{(i)} = \theta^{(i-1)}$
        \EndIf
    \While {$Tuning$}
        \State $\sigma_e^2 = 2.4^2 Var(\theta^{(a-tune)}, \dots, \theta^{(a)}|D)/d$
    \EndWhile
    \State $i+=1$
\EndWhile
\end{algorithmic}
\end{algorithm}

\section{Results}

\subsection{Data Reduction}

\begin{figure}[b!]
    \centering
    \subfigure{\includegraphics[width=0.4\textwidth]{../data/unsmoothed/5_kw_m2/plots/1903-02_05}}
    \qquad
    \subfigure{\includegraphics[width=0.4\textwidth]{../data/smoothed/5_kw_m2/plots/1903-02_05}}
    \caption{Unsmoothed and smoothed temperature data for one DFT test}
    \label{fig:tc_data}
\end{figure}

\begin{figure}
    \centering
    \includegraphics[width=0.6\textwidth]{../data/residuals/5_kw_m2/1903-02_05}
    \caption{Residuals from the Gaussian process fit}
    \label{fig:gp_residuals}
\end{figure}

\section{Results}

\begin{figure}[b!]
    \centering
    \subfigure{\includegraphics[width=0.4\textwidth]{../figures/alpha_hist_0_dft_5_kwm2_1}}
    \qquad
    \subfigure{\includegraphics[width=0.4\textwidth]{../figures/alpha_trace_0_dft_5_kwm2_1}}
    \caption{Histogram and trace plots for thermal conductivity of insulation, $k^{(1)}_{ins}$}
    \label{fig:param_trace_1}
\end{figure}

\begin{figure}[b!]
    \centering
    \subfigure{\includegraphics[width=0.4\textwidth]{../figures/sigma_hist_dft_5_kwm2_1}}
    \qquad
    \subfigure{\includegraphics[width=0.4\textwidth]{../figures/sigma_trace_dft_5_kwm2_1}}
    \caption{Histogram and trace plots for $\sigma_q^2$}
    \label{fig:sigma_trace_1}
\end{figure}

\begin{figure}
    \centering
    \includegraphics[width=0.8\textwidth]{../figures/calibrated_results_dft_5_kwm2_1}
    \caption{Calibrated results assuming $k^{(1)}_{ins}(T) = k^{(1)}_0$}
    \label{fig:cal_results_1}
\end{figure}

\begin{figure}[b!]
    \centering
    \subfigure{\includegraphics[width=0.4\textwidth]{../figures/alpha_hist_0_dft_5_kwm2_2}}
    \qquad
    \subfigure{\includegraphics[width=0.4\textwidth]{../figures/alpha_trace_0_dft_5_kwm2_2}}

    \subfigure{\includegraphics[width=0.4\textwidth]{../figures/alpha_hist_1_dft_5_kwm2_2}}
    \qquad
    \subfigure{\includegraphics[width=0.4\textwidth]{../figures/alpha_trace_1_dft_5_kwm2_2}}
    \caption{Histogram and trace plots for thermal conductivity of insulation, $k^{(2)}_{ins}$. Top is $k^{(2)}_0$ and bottom is $k^{(2)}_1$.}
    \label{fig:param_trace_2}
\end{figure}

\begin{figure}
    \centering
    \includegraphics[width=0.6\textwidth]{../figures/alpha_correlation}
    \caption{Scatter plot showing the correlation between $k^{(2)}_0$ and $k^{(2)}_1$.}
    \label{fig:alpha_correlation}
\end{figure}

\begin{figure}[b!]
    \centering
    \subfigure{\includegraphics[width=0.4\textwidth]{../figures/sigma_hist_dft_5_kwm2_2}}
    \qquad
    \subfigure{\includegraphics[width=0.4\textwidth]{../figures/sigma_trace_dft_5_kwm2_2}}
    \caption{Histogram and trace plots for $\sigma_q^2$}
    \label{fig:sigma_trace_2}
\end{figure}

\begin{figure}
    \centering
    \includegraphics[width=0.8\textwidth]{../figures/calibrated_results_dft_5_kwm2_2}
    \caption{Calibrated results assuming $k^{(2)}_{ins}(T) = k^{(2)}_0 + k^{(2)}_1 T$}
    \label{fig:cal_results_2}
\end{figure}


\section{Conclusions}



\acknowledgements

This work is supported by the JB (Java Bean) Association under Award No. 8675309.

%% References with bibTeX database:
\bibliographystyle{Bibliography_Style}
\scriptsize{
\bibliography{../../../../dissertation-experimentation/References}
}

\end{document}
